\chapter{Introdução}\label{cap:introducao}

O câncer de mama permanece como a principal causa de mortalidade por câncer entre mulheres em todo o mundo, sendo responsável por um grande número de óbitos em diversos países \cite{cadrin2023unleashing,SECHOPOULOS2021214}. Apesar dos avanços em métodos de triagem e tratamento, a detecção precoce ainda enfrenta desafios consideráveis, como a variabilidade na interpretação dos exames por profissionais de saúde e a ocorrência de falsos positivos, que podem gerar ansiedade e procedimentos desnecessários para as pacientes.

Nesse cenário, a inteligência artificial (IA) tem se mostrado uma ferramenta promissora no apoio ao diagnóstico do câncer de mama. A utilização de redes neurais convolucionais (CNN), por exemplo, para a detecção de anomalias de forma automática, juntamente com uma avaliação médica, contribui para evitar procedimentos desnecessários, minimizar o impacto psicológico nas pacientes e otimizar os recursos do sistema de saúde \cite{SECHOPOULOS2021214,cadrin2023unleashing}.

Para que seja possível utilizar redes neurais na detecção de anormalidades em um exame de mama, é necessário treiná-las previamente. Esse processo envolve fornecer uma grande quantidade de imagens já classificadas, permitindo que a rede aprenda a reconhecer padrões específicos associados. O intuito é de que após esse treinamento, a rede seja capaz de analisar uma nova imagem e fazer uma previsão com base no conhecimento adquirido. Entretanto, treinar redes neurais do zero exige muitos dados rotulados e alto poder computacional. Para contornar essas limitações, utiliza-se o \textit{Transfer Learning}, que reaproveita redes já treinadas em grandes bases, permitindo bons resultados mesmo com poucos dados e infraestrutura reduzida \cite{ISIN2017268}.


Além disso, as imagens precisam passar por um pré-processamento. Essa etapa é necessária para que as imagens estejam em um formato e qualidade que
facilite a extração de informações relevantes pelos algoritmos. Uma das técnicas utilizadas nesse processo é a Transformada \textit{Wavelet}, que permite decompor a imagem em níveis de resoluções diferentes \cite{leite2018analise}. Isso facilita a identificação de detalhes sutis, como microcalcificações e
bordas de nódulos, ao mesmo tempo em que reduz ruídos indesejados. Dessa forma, o uso da \textit{Wavelet} contribui para tornar os dados mais adequados ao aprendizado da rede neural, aumentando a eficiência na detecção de padrões
associados ao câncer de mama.

Diante desse cenário, esta pesquisa tem como objetivo treinar tanto uma arquitetura convolucional simples, desenvolvida do zero, quanto modelos baseados em \textit{Transfer Learning} para a detecção de anomalias em imagens de mamografia utilizando uma base de dados pública. Este trabalho de conclusão de curso compara o desempenho dessas abordagens com e sem a aplicação da Transformada \textit{Wavelet} como técnica de pré-processamento, além de avaliar o impacto do uso de \textit{data augmentation}, entendida como o conjunto de transformações aplicadas às imagens (como rotações, espelhamentos e variações de escala) com o objetivo de aumentar a diversidade dos dados disponíveis para treinamento. A proposta é verificar se a aplicação da \textit{Wavelet} e das técnicas de aumento de dados contribui para melhorar a acurácia dos modelos, facilitando a identificação de padrões relevantes e auxiliando na detecção precoce do câncer de mama.

\section{Objetivos gerais}
Desenvolver e comparar algoritmos para detecção automática de nódulos mamários em mamografias, integrando a decomposição \textit{Wavelet} multinível como técnica de pré-processamento e diferentes arquiteturas de redes neurais convolucionais, incluindo modelos treinados do zero e com \textit{Transfer Learning}, implementados em Python. O objetivo é investigar se a aplicação da \textit{Wavelet} e do \textit{data augmentation} contribui para elevar a precisão diagnóstica, avaliando o impacto de cada estratégia no desempenho das redes.
\section{Objetivos específicos}

\begin{itemize}
    
    \item Selecionar e pré-processar imagens de mamografia provenientes da base pública \textit{Mini-MIAS}.
    
    \item Implementar um algoritmo de decomposição \textit{Wavelet} para destacar padrões relevantes nas imagens e reduzir ruídos.
    
    \item Aplicar técnicas de \textit{data augmentation}, compreendidas como transformações geométricas e fotométricas que aumentam a variabilidade das imagens disponíveis para o treinamento.
    
    \item Treinar redes neurais convolucionais (CNNs) nas arquiteturas ResNet-18, ResNet-34 e ResNet-50 com o uso de \textit{Transfer Learning}.
    
    \item Avaliar o desempenho do modelo em diferentes cenários de entrada (com e sem \textit{Wavelet}, com e sem \textit{data augmentation}), utilizando métricas como acurácia, precisão, recall e F1-score.
    
    \item Comparar os resultados obtidos para verificar o impacto das estratégias de pré-processamento e aumento de dados na detecção automática de nódulos mamários.

\end{itemize}



% \section{Organização do texto}

% O texto está organizado da seguinte forma: No \autoref{cap:revisao} é apresentado um pouco mais de como fazer um outro capítulo, apresentando ainda formas para inserir figuras. No \autoref{cap:proposta} é apresentado uma forma para adicionar uma tabela. Por fim, no \autoref{cap:conclusoes} são apresentadas as conclusões sobre este trabalho.